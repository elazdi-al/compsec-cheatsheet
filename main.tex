\documentclass[8pt]{article}
\usepackage[top=0.2in, bottom=0.2in, left=0.1in, right=0.1in]{geometry} % Adjust margins individually
\usepackage{amsmath,amssymb,amsthm}
\usepackage{enumitem}
\usepackage{graphicx} % Required for \includegraphics
\usepackage{ragged2e}
             
\usepackage[absolute,overlay]{textpos}
\usepackage{array} % For better column definitions
\usepackage{booktabs} % For improved table aesthetics
\raggedbottom
\setlist[itemize]{leftmargin=1em}
\newcommand{\dist}[2]{\left\langle #1,\, #2 \right\rangle}
\setlength{\TPHorizModule}{0.1mm} % Set horizontal units to millimeters
\setlength{\TPVertModule}{0.1mm}  % Set vertical units to millimeters
\usepackage{multicol}
\setlength{\parindent}{0pt}
\setlength{\parskip}{5pt plus 1pt minus 1pt}
\usepackage{setspace}
\setstretch{1}
\pagestyle{empty}
\setlength{\parskip}{0pt}      % space between paragraphs
\setlength{\parindent}{0pt}    % optional: remove indentation
\setlist{nosep, topsep=0pt, partopsep=0pt, parsep=0pt, itemsep=0pt}
% Redefine the textblock environment
\let\originaltextblock\textblock
\let\endoriginaltextblock\endtextblock

\renewenvironment{textblock}[2][]{%
    \originaltextblock[#1]{#2}%
    \fcolorbox{red}{white}{%
    \begin{minipage}{#2}%
}{%
    \end{minipage}%
    }%
    \endoriginaltextblock
}


\begin{document}
\begin{titlepage}
	\centering
	\vspace*{1cm}
	{\Huge \textbf{Computer Security - CheatSheet}} \\
	\vspace{20px}
	{\LARGE IN~BA5 - Martin Werner Licht} \\
	\vspace*{1cm}
	{\Large Notes by Ali EL AZDI} \\
	\vfill

	\begin{justify}

	\end{justify}
	\vspace*{100px}

	{\large October 26th, 2025}
	\vspace*{20px}
\end{titlepage}


\vspace*{-20px}
\noindent
\begin{minipage}[t]{0.49\textwidth}
	\noindent\textbf{CompSec Properties}
	\begin{itemize}
		\item[-] \textbf{Confidentiality}. prevention of unauthorized disclosure of information.
		\item[-] \textbf{Integrity.} prevention of unauthorized modification of information.
		\item[-] \textbf{Availability.} prevention of unauthorized denial of service or access to information and resources.
		\item[-] \textbf{Authenticity.} assurance that entities (users, systems, or data) are genuine and can be verified as such.
		\item[-] \textbf{Anonymity.} protection of an individual's identity from being disclosed or linked to specific actions or data.
		\item[-] \textbf{Non-repudiation.} assurance that a party in a communication cannot deny the authenticity of their signature or the sending of a message.
	\end{itemize}

	\noindent \textbf{Threat Model.} describes the ressources available to the adversary and their capabilities \textit{(has access to internet, but doesn't have access to the internal network of the company.)}\\
	\noindent \textbf{Threat.} Who might attack which assets, using what resources, with what goal, how, and with what probability\\
	\noindent \textbf{Vulnerability.} Specific weakness that could be exploited by adversaries with interest in a lot of different assets \textit{(API is not protected, password appears in plain text\dots)}\\
	\noindent \textbf{Harm.} The bad thing that could happen when the \textbf{threat} materializes. \textit{(adversary steals the money, learns my password\dots)}
\end{minipage}
\hfill
\noindent
\begin{minipage}[t]{0.49\textwidth}
	\noindent \textbf{Security Policy}\\
	A high level description of the security properties that must hold in the system in relation to assets and principals
	\begin{itemize}
		\item[-] \textbf{Assets (objects).} anything with value (data, files , memory) that needs to be protected
		\item[-] \textbf{Principals (subjects).} people, computer programs, services
	\end{itemize}
	\textbf{Examples}\\
	\textit{Confidentiality.} authorized users may read a file\\
	\textit{Integrity.} authorized programs may write a file\\
	\textit{Availability.} authorized services can access a file\\
	\noindent\textbf{The Adversary.} malicious entity aiming at breaching the security policy and \textbf{will} choose the optimal way to use her ressources to mount an attack that violates the security properties.\\
	\noindent \textbf{Security Mechanism.} Technical mechanism used to ensure that the security policy is not violated by an adversary within the threat model, \textbf{we can only prepare for threats we're aware of} \\\textit{(Policy. ensure messages cannot be read by anyone but the sender and the receiver, Mechanism. encrypt the message before sending)}
	\noindent \textbf{Composition of Security Mechanisms}
	\begin{itemize}
		\item[-] \textbf{Defence in depth.} As long as one remains unbroken the Security Policy isn't broken) \textit{(two-factor auth)}
		\item[-] \textbf{Weakest Link.} (if anyone fails the \textbf{Security Policy} is broken) security questions in case of a lost password, no need to know the password but just the answer
	\end{itemize}
\end{minipage}
\newpage
% Reduce paragraph spacing/indentation:
\setlength{\parindent}{0pt}
\setlength{\parskip}{0pt}



\footnotesize  % Slightly smaller font to fit all content comfortably

\begingroup
% ----------------------------------------------------------------------
% Adjust spacing for displayed equations:
\setlength{\abovedisplayskip}{3pt}
\setlength{\belowdisplayskip}{3pt}
\setlength{\abovedisplayshortskip}{2pt}
\setlength{\belowdisplayshortskip}{2pt}

% Adjust spacing for itemize/enumerate:
\let\oldenumerate\enumerate
\def\enumerate{\oldenumerate
	\setlength{\itemsep}{0pt}
	\setlength{\parskip}{0pt}
	\setlength{\parsep}{0pt}
}
\let\olditemize\itemize
\def\itemize{\olditemize
	\setlength{\itemsep}{0pt}
	\setlength{\parskip}{0pt}
	\setlength{\parsep}{0pt}
}
\setstretch{0.6}
% ----------------------------------------------------------------------
\small
\begin{multicols}{2}

	\noindent \textbf{Piecewise Continuity \& Differentiability}\\
	$f:[a,b]\to\mathbb{R}$ is \emph{piecewise continuous} if there is a partition
	\[
		a=a_0 < a_1 < \dots < a_n = b
	\]
	such that $\lim_{x \to a_i^-} f(x)$ and $\lim_{x \to a_i^+} f(x)$ exist (finite).\\
	Similarly, $f$ is \emph{piecewise $C^1$} if it is continuously differentiable on each open subinterval and the one‐sided derivatives at boundaries exist.\\

	\noindent \textbf{Euler's Formulas}
	\[
		e^{x+iy} = e^{x}\bigl(\cos y + i \sin y\bigr),\;
		\sin x = \frac{e^{ix}-e^{-ix}}{2i},\;
		\cos x = \frac{e^{ix}+e^{-ix}}{2}.
	\]

	\noindent \textbf{Orthogonality (Sine/Cosine Products)}\\
	For $n,m \in \mathbb{N}_{\ge 1}$ and period $T>0$:
	\[
		\frac{2}{T} \int_{0}^{T}
		\cos\!\Bigl(\tfrac{2\pi n}{T}x\Bigr)\cos\!\Bigl(\tfrac{2\pi m}{T}x\Bigr)\,dx
		=
		\begin{cases}
			1 & n=m,    \\
			0 & n\neq m
		\end{cases}
	\]
	(Same for $\sin\sin$, and $\cos\sin$ integrates to $0$.)\\

	\noindent \textbf{Integration Over One Period}\\
	If $f$ is $T$-periodic and piecewise continuous, then for any $a\in\mathbb{R}$:
	\[
		\int_{a}^{a+T} f(x)\,dx = \int_{0}^{T} f(x)\,dx.
	\]

	\noindent \textbf{Dirichlet's Theorem (Pointwise Convergence)}\\
	Let \( f : \mathbb{R} \to \mathbb{R} \) be \( T \)-periodic and piecewise \( C^1 \). Then, for all \( x \in \mathbb{R} \),
	\[
		Ff(x) = \lim_{t \to 0} \frac{f(x - t) + f(x + t)}{2}.
	\]

	\noindent \textbf{Real Fourier Series}\\
	For $f:\mathbb{R}\to\mathbb{R}$, $T$-periodic, piecewise $C^1$, the real Fourier series is \vspace{-3px}
	\[ \vspace{-8px}
		Ff(x)= \frac{a_{0}}{2}
		+ \sum_{n=1}^{\infty}\!\Bigl[a_{n}\cos\!\bigl(\tfrac{2\pi n}{T}x\bigr)
			+ b_{n}\sin\!\bigl(\tfrac{2\pi n}{T}x\bigr)\Bigr].
	\]
	\noindent \textbf{Fourier Coefficients:} \vspace{-3px}
	\[
		a_{n} = \frac{2}{T}\!\int_{0}^{T}\!f(x)
		\cos\!\Bigl(\tfrac{2\pi n}{T}x\Bigr)\,dx,\quad
		b_{n} = \frac{2}{T}\!\int_{0}^{T}\!f(x)
		\sin\!\Bigl(\tfrac{2\pi n}{T}x\Bigr)\,dx,
	\] \vspace{-8px}
	\[
		a_{0} = \frac{2}{T}\int_{0}^{T} f(x)\,dx.
	\]
	\noindent \textbf{Parity:} If $f$ is even, $b_{n}=0$; if $f$ is odd, $a_{n}=0$.\\

	\small

	\noindent \textbf{Term-by-Term Differentiation}\\
	If $f$ is $T$-periodic, continuous, and piecewise $C^1$, then
	\[
		\frac{d}{dx}\bigl[Ff(x)\bigr]
		= \sum_{n=1}^\infty \frac{2\pi n}{T}\bigl[-\,a_n\sin(\tfrac{2\pi n}{T}x)
			+b_n\cos(\tfrac{2\pi n}{T}x)\bigr]
	\]
	\[
		= \lim_{t \to 0} \frac{f'(x - t) + f'(x + t)}{2}.
	\]
	\noindent \textbf{Term-by-Term Integration}\\
	If $f$ is $T$-periodic, continuous, and piecewise $C^1$, then
	\[
		\int Ff(x)\,dx
		= \sum_{n=1}^\infty \frac{T}{2 n \pi} \left[ a_n \sin\left(\tfrac{2\pi n}{T}x\right)
			- b_n \cos\left(\tfrac{2\pi n}{T}x\right) \right] + C
	\]
	\[
		= \lim_{h \to 0} \frac{1}{2h} \int_{x - h}^{x + h} Ff(t)\,dt,
	\]
	where $C$ is the constant of integration.

	\noindent \textbf{Poisson on }$[a,b]$
	\[
		\begin{cases}
			-\,u''(x)=f(x), \\[2pt]
			u(a)=g_a,\quad u(b)=g_b,
		\end{cases}
		\quad
		L = b - a
	\]
	\[
		u^g(x)=\frac{g_b-g_a}{b-a}\,x \;+\;\frac{b\,g_a - a\,g_b}{b-a}.
	\]
	\[
		f(x)=\sum_{n=1}^{\infty} b_n \,\sin\!\Bigl(\frac{n\pi x}{L}\Bigr),
		\quad
		b_n=\frac{2}{L}\int_{0}^{L}f(t)\,\sin\!\Bigl(\frac{n\pi t}{L}\Bigr)\,dt,
	\]
	\[
		u^f(x)=\sum_{n=1}^{\infty} b_n \,\frac{L^2}{\pi^2\,n^2}
		\,\sin\!\Bigl(\frac{n\pi x}{L}\Bigr).
	\]
	\[
		u(x)=u^g(x)+u^f(x) \quad (\text{superposition principle}).
	\]
	\noindent \textbf{Poisson with mass term on }$\mathbb{R}$ \vspace{-5px}
	\[
		-\,u''(x)+k^2\,u(x)=f(x),
		\quad
		\widehat{u}(\alpha)=\frac{\widehat{f}(\alpha)}{\alpha^2 + k^2}.
	\]
	\vspace{-5px}
	\[
		g(x)=\sqrt{\frac{\pi}{2}} \frac{1}{k}\,e^{-k|x|},
		\quad
		\widehat{g}(\alpha)=\frac{1}{\alpha^2 + k^2},
	\]
	\vspace{-5px}\[
		u(x)=(g*f)(x)
		=\frac{1}{2k}\!\int_{-\infty}^{\infty} f(y)\,e^{-k|\,x-y\,|}\,dy.
	\]

	\columnbreak

	\noindent \textbf{Complex Fourier Coefficient}\\
	Let \( f : \mathbb{R} \to \mathbb{R} \) be \( T \)-periodic and piecewise continuous.  The complex Fourier coefficients are:
	\[
		c_n = \frac{1}{T} \int_0^T f(x)\, e^{-i \frac{2\pi}{T} n x} \, dx,
		\quad
		Ff(x) = \sum_{n=-\infty}^{\infty} c_{n}\,e^{i\frac{2\pi n}{T}x}.\vspace{-10px}
	\]
	For \( \phi : \mathbb{R} \to \mathbb{C} \), \vspace{-5px}
	\[ \int_a^b \phi(x)\,dx
		= \int_a^b \operatorname{Re}(\phi(x)) \, dx + i \int_a^b \operatorname{Im}(\phi(x)) \, dx. \vspace{-5px}
	\]
	\noindent \textbf{Relation to \( (a_n, b_n) \)}\quad
	\[ \vspace{-5px}
		c_{n} = \tfrac{1}{2}(a_{n} - i\,b_{n}),\
		c_{-n} = \tfrac{1}{2}(a_{n} + i\,b_{n}),\
		c_{0} = \tfrac{a_{0}}{2}.\]\[
		a_{n} = c_n + c_{-n}\
		a_{0} = 2 c_0 \
		b_n = \operatorname{Re}(c_{-n} - c_n)
	\]

	\noindent \textbf{Fourier Series on $[0,L]$}\\
	For $f:[0,L]\to \mathbb{R}$ (piecewise $C^1$):
	\[
		F_{c}f(x)=\tfrac{\tilde a_{0}}{2} + \sum_{n=1}^{\infty}\!\tilde a_{n}\cos\!\Bigl(\tfrac{\pi n}{L}x\Bigr),
		\quad \tilde a_{n}=\frac{2}{L}\int_{0}^{L} f(x)\cos\!\Bigl(\tfrac{\pi n}{L}x\Bigr)\,dx.
	\]
	\[
		F_{s}f(x)=\sum_{n=1}^{\infty}\!\tilde b_{n}\sin\!\Bigl(\tfrac{\pi n}{L}x\Bigr),
		\quad \tilde b_{n}=\frac{2}{L}\int_{0}^{L} f(x)\sin\!\Bigl(\tfrac{\pi n}{L}x\Bigr)\,dx.
	\]

	\noindent \textbf{Parseval's Identity (Periodic Case)}\\
	If $f$ is $T$-periodic (piecewise $C^1$),
	\[
		\frac{2}{T}\int_{0}^{T} f^{2}(x)\,dx
		= \frac{a_{0}^{2}}{2} + \sum_{n=1}^{\infty}(a_{n}^{2}+b_{n}^{2})
		= 2\sum_{n=-\infty}^{\infty}\!\!\bigl|c_{n}\bigr|^{2}.
	\]

	\noindent \textbf{Plancherel Theorem}\\
	Let $f \in L^2(\mathbb{R})$. Then its Fourier transform $\hat{f}$ is also in $L^2(\mathbb{R})$, and:
	\[
		\int_{-\infty}^\infty |f(x)|^2 \, dx
		= \int_{-\infty}^\infty |\hat{f}(\xi)|^2 \, d\xi.
	\]

	\noindent \textbf{The Fourier Transform}\\
	If $f:\mathbb{R}\to\mathbb{R}$ with $\int_{-\infty}^{\infty}\!|f(x)|\,dx<\infty$, its (unitary) Fourier transform is
	\[
		\widehat{f}(\alpha)=\frac{1}{\sqrt{2\pi}}
		\int_{-\infty}^{\infty} f(x)\, e^{-\,i\,\alpha x}\,dx, \vspace{-10px}
	\]
	\noindent \textbf{Inverse Transform}\\
	If $\varphi(\alpha)$ is similarly integrable,
	\[
		\mathcal{F}^{-1}(\varphi)(x)= \frac{1}{\sqrt{2\pi}}
		\int_{-\infty}^{\infty}\!\varphi(\alpha)\,e^{\,i\,\alpha x}\,d\alpha.
	\]
	\noindent \textbf{Convolution Product}\\
	Let \(f,g:\mathbb{R} \to \mathbb{R}\) such that
	\(\int_{-\infty}^{+\infty} |f(x)|\,dx < +\infty\),
	\(\int_{-\infty}^{+\infty} |g(x)|\,dx < +\infty\). \vspace{-5px}
	\[
		(f * g)(x)
		= \int_{-\infty}^{+\infty} f(x - t)\,g(t)\,dt
		= \int_{-\infty}^{+\infty} f(t)\,g(x - t)\,dt
	\]

	\vfill
	\setlength{\columnsep}{-70pt} % Adjusted for better column fitting
	\hspace{-15px}
	\normalsize
	\begin{minipage}[htp]{0.55\textwidth}
		\begin{multicols}{2}
			\textbf{Scaling} \\
			$\mathcal{F}\{f(ax)\} = \frac{1}{|a|} \hat{f}\!\bigl(\tfrac{\alpha}{a}\bigr)$ \\[4pt]

			\textbf{Shifting} \\
			$\mathcal{F}\{f(x - x_0)\} = e^{-i\alpha x_0}\,\hat{f}(\alpha)$ \\[4pt]

			\textbf{Modulation} \\
			$\mathcal{F}\{e^{i\omega_0 x} f(x)\} = \hat{f}(\alpha - \omega_0)$ \\[4pt]

			\textbf{Convolution} \\
			$\mathcal{F}[f * g](\alpha)
				= \sqrt{2\pi} \,\hat{f}(\alpha)\,\hat{g}(\alpha)$ \\[4pt]
			\columnbreak

			\textbf{Differentiation} \\[2pt]
			$\mathcal{F}\!\Bigl\{\dfrac{d^n}{dx^n}f(x)\Bigr\} = (i\alpha)^n \hat{f}(\alpha)$ \\[4pt]

			\textbf{Integration} \\
			$\mathcal{F}\!\Bigl\{\int_{-\infty}^x f(\xi)\,d\xi\Bigr\}
				= \dfrac{1}{i\alpha}\,\hat{f}(\alpha), \ \alpha \neq 0$ \\[4pt]

			\textbf{Differentiation of Transform} \\[2pt]
			$
				\mathcal{F}\big((-ix)^n f(x)\big)(\alpha) = \frac{\partial^n}{\partial \alpha^n} \hat{f}(\alpha) \\[4pt]
			$\\
			\textbf{Multiplication} \\
			$
				\mathcal{F}\{f(x) \cdot g(x)\} = \sqrt{2\pi}\left( \mathcal{F}\{f(x)\} * \mathcal{F}\{g(x)\} \right)
			$ \\[4pt]
		\end{multicols}
	\end{minipage}
	\normalsize
	\noindent \textbf{Important Trigonometric Identities} \\[2pt]
	$\sin(2x) = 2\sin x\cos x,$ \\[3pt]
	$\cos(2x) = 2\cos^2 x - 1 = 1 - 2\sin^2 x = \cos^2 x - \sin^2 x,$ \\[3pt]
	$\cos(a \pm b) = \cos a\cos b \mp \sin a\sin b,$ \\[3pt]
	$\sin(a \pm b) = \sin a\cos b \pm \cos a\sin b,$ \\[3pt]
	$\cos a\cos b = \tfrac{1}{2}\bigl[\cos(a - b) + \cos(a + b)\bigr],$ \\[3pt]
	$\sin a\sin b = \tfrac{1}{2}\bigl[\cos(a - b) - \cos(a + b)\bigr],$ \\[3pt]
	$\sin a\cos b = \tfrac{1}{2}\bigl[\sin(a + b) + \sin(a - b)\bigr],$ \\[3pt]
	$\cos a\sin b = \tfrac{1}{2}\bigl[\sin(a + b) - \sin(a - b)\bigr].$\\[3pt]
	$cos(n \pi) = (-1)^n$
	\normalsize
\end{multicols}
\endgroup


\end{document}

